%!TEX root = ../template.tex
%%%%%%%%%%%%%%%%%%%%%%%%%%%%%%%%%%%%%%%%%%%%%%%%%%%%%%%%%%%%%%%%%%%
%% chapter1.tex
%% NOVA thesis document file
%%
%% Chapter with introduction
%%%%%%%%%%%%%%%%%%%%%%%%%%%%%%%%%%%%%%%%%%%%%%%%%%%%%%%%%%%%%%%%%%%

\typeout{NT FILE chapter1.tex}%

\chapter{Introduzione}
\label{chapt:1}

\prependtographicspath{{Chapters/Figures/Covers/}}

% epigraph configuration
\epigraphfontsize{\small\itshape}
\setlength\epigraphwidth{12.5cm}
\setlength\epigraphrule{0pt}

\section{Informazioni di Base}

\subsection{Definizione e Importanza del TSP}

L'importanza del \gls{TSP} va oltre la sua formulazione apparentemente semplice, ramificandosi in vari campi e applicazioni. In logistica e trasporti, le
soluzioni al \gls{TSP} consentono un routing ottimale in grado di portare a
significativi risparmi di costi e miglioramenti di efficienza. Nella
produzione, gli algoritmi \gls{TSP} vengono utilizzati per ottimizzare i
movimenti di trapani nella produzione di circuiti stampati o di bracci
robotici nelle linee di assemblaggio per ridurre i tempi di produzione.
Oltre a questi, il quadro del \gls{TSP} viene impiegato nel sequenziamento del
DNA, nell'astronomia e persino nella creazione di opere d'arte, mostrando
la sua versatilità e importanza nel risolvere problemi del mondo reale.

\subsection{Panoramica Storica}

Il Travelling Salesman Problem (\gls{TSP}) rappresenta una delle sfide più studiate nella matematica computazionale e nell'ottimizzazione, con una storia che si intreccia con lo sviluppo della ricerca matematica e operativa. La sua evoluzione dalle prime esplorazioni concettuali a un problema cruciale nella progettazione degli algoritmi è una testimonianza della sua complessità e della sua ampia applicabilità.
Lo studio del \gls{TSP} può essere fatto risalire ai lavori di Sir William Rowan Hamilton e Thomas Penyngton Kirkman nel XIX secolo, incentrati sui cicli e sui percorsi hamiltoniani. Karl Menger, negli anni '20, ha introdotto il "Problema del Messaggero", gettando le basi per la formulazione moderna del \gls{TSP} e sottolineando la ricerca di un routing efficiente.

L'avvento dei computer e delle tecniche di ricerca operativa ha segnato una fase significativa nella ricerca sul \gls{TSP}. I primi pionieri come George Dantzig e Delbert Ray Fulkerson hanno applicato la programmazione lineare e i metodi dei piani di taglio, illustrando la complessità computazionale del problema e il suo potenziale per le applicazioni pratiche.

Negli anni '40, il \gls{TSP} ha iniziato a trovare rilevanza in agricoltura e statistica, con i ricercatori che applicavano il quadro del problema per ottimizzare i processi di rilevamento e raccolta dei dati, mostrando la versatilità del \gls{TSP} oltre la matematica teorica.

Negli ultimi decenni, l'esplorazione di algoritmi euristici e bio-ispirati, in particolare il lavoro di Marco Dorigo sull'ottimizzazione della colonia di formiche, ha aperto nuove vie per risolvere il \gls{TSP}, evidenziando l'innovazione continua nell'affrontare la sua NP-hardness.


\section{Il Problema}

\textbf{Il Problema del Viaggiatore:} Data una serie di città e le distanze tra ogni coppia di città, trovare il percorso più breve possibile che visiti ogni città esattamente una volta e torni alla città originale.

\subsection{Formulazione Matematica}

Dato un grafo \(G = (V, E)\), dove \(V\) è l'insieme dei vertici (città) e \(E\) è l'insieme degli archi (percorsi tra le città), con ogni arco \((i, j) \in E\) assegnato un peso \(w_{ij}\) che rappresenta il costo di viaggio dalla città \(i\) alla città \(j\), il \gls{TSP} cerca un ciclo hamiltoniano di peso minimo.
La funzione obiettivo da minimizzare è:
\[
	\min \sum_{i=1}^{n} \sum_{j \ne i, j=1}^{n} w_{ij} x_{ij},
\]
soggetta ai vincoli:
\[
	\sum_{i=1, i \ne j}^{n} x_{ij} = 1, \quad \forall j \in V,
\]
\[
	\sum_{j=1, j \ne i}^{n} x_{ij} = 1, \quad \forall i \in V,
\]
insieme ai vincoli di eliminazione dei sottoitinerari, assicurando che ogni città sia visitata esattamente una volta e il tour sia ciclico.

\subsection{Formulazioni di Programmazione Lineare Intera}

\subsubsection{Formulazione di Miller–Tucker–Zemlin}

La formulazione \gls{MTZ} introduce variabili ausiliarie \(u_i\) per l'ordinamento, insieme alle variabili binarie \(x_{ij}\):

Obiettivo:
\[
	\min \sum_{i=1}^{n} \sum_{j \ne i, j=1}^{n} c_{ij} x_{ij},
\]
Soggetto a:
\[
	\sum_{i=1, i \ne j}^{n} x_{ij} = 1, \quad \forall j,
\]
\[
	\sum_{j=1, j \ne i}^{n} x_{ij} = 1, \quad \forall i,
\]
\[
	u_i - u_j + 1 \le (n-1)(1 - x_{ij}), \quad \forall 2 \le i \ne j \le n,
\]
\[
	2 \le u_i \le n, \quad \forall 2 \le i \le n.
\]

Questa formulazione impone un singolo tour attraverso variabili ausiliarie di ordinamento, prevenendo efficacemente i sottoitinerari.

\subsubsection{Formulazione di Dantzig–Fulkerson–Johnson}

La formulazione DFJ, nota per la sua efficienza, introduce direttamente i vincoli di eliminazione dei sottoitinerari:

Obiettivo:
\[
	\min \sum_{i=1}^{n} \sum_{j \ne i, j=1}^{n} c_{ij} x_{ij},
\]
Soggetto a:
\[
	\sum_{i=1, i \ne j}^{n} x_{ij} = 1, \quad \forall j,
\]
\[
	\sum_{j=1, j \ne i}^{n} x_{ij} = 1, \quad \forall i,
\]
\[
	\sum_{i \in S, j \notin S} x_{ij} \ge 1, \quad \forall S \subset V, S \ne \emptyset, S \ne V.
\]


Questa formulazione combatte i sottoitinerari assicurando che almeno un bordo esca da ogni sottinsieme di vertici, garantendo un singolo tour connesso.

% TODO: Fix this

% \begin{table}[h!]
% 	\centering
% 	\begin{tabularx}{\textwidth}{|X|X|X|}
% 			\hline
% 			\textbf{Caratteristica} & \textbf{Miller-Tucker-Zemlin (MTZ)} & \textbf{Dantzig-Fulkerson-Johnson (DFJ)} \\
% 			\hline
% 			\textbf{Funzione Obiettivo} & \(\min \sum_{i=1}^{n} \sum_{j \ne i, j=1}^{n} c_{ij} x_{ij}\) & \(\min \sum_{i=1}^{n} \sum_{j \ne i, j=1}^{n} c_{ij} x_{ij}\) \\
% 			\hline
% 			\textbf{Vincoli} & 
% 			\begin{tabular}{@{}l@{}}
% 			\(\sum_{i=1, i \ne j}^{n} x_{ij} = 1, \ \forall j\) \\
% 			\(\sum_{j=1, j \ne i}^{n} x_{ij} = 1, \ \forall i\) \\
% 			\(u_i - u_j + 1 \le (n-1)(1 - x_{ij}), \ \forall 2 \le i \ne j \le n\) \\
% 			\(2 \le u_i \le n, \ \forall 2 \le i \le n\)
% 			\end{tabular} & 
% 			\begin{tabular}{@{}l@{}}
% 			\(\sum_{i=1, i \ne j}^{n} x_{ij} = 1, \ \forall j\) \\
% 			\(\sum_{j=1, j \ne i}^{n} x_{ij} = 1, \ \forall i\) \\
% 			\(\sum_{i \in S, j \notin S} x_{ij} \ge 1, \ \forall S \subset V, S \ne \emptyset, S \ne V\)
% 			\end{tabular} \\
% 			\hline
% 			\textbf{Punti di Forza} & 
% 			\begin{tabular}{@{}l@{}}
% 			Riduce il problema a un singolo tour \\
% 			Previene efficacemente i sottoitinerari
% 			\end{tabular} & 
% 			\begin{tabular}{@{}l@{}}
% 			Include direttamente i vincoli di eliminazione dei sottoitinerari \\
% 			Efficiente per problemi di dimensioni piccole e medie
% 			\end{tabular} \\
% 			\hline
% 			\textbf{Limitazioni} & 
% 			\begin{tabular}{@{}l@{}}
% 			Numero elevato di variabili e vincoli \\
% 			Computazionalmente intensivo per problemi grandi
% 			\end{tabular} & 
% 			\begin{tabular}{@{}l@{}}
% 			Meno efficace per istanze molto grandi \\
% 			I vincoli possono essere ridondanti o costosi computazionalmente
% 			\end{tabular} \\
% 			\hline
% 	\end{tabularx}
% 	\caption{Confronto Trasposto delle Formulazioni del Problema del Viaggiatore (TSP)}
% 	\label{tab:tsp_formulations_transposed}
% \end{table}


\subsection{Varianti e Applicazioni}

Il \gls{TSP} si adatta a un'ampia gamma di sfide pratiche, dimostrando la sua versatilità. Le varianti chiave includono:

\begin{itemize}
	\item \textbf{TSP Simmetrico vs. Asimmetrico (STSP vs. ATSP):} STSP presuppone costi di viaggio identici tra qualsiasi coppia di città, mentre ATSP consente costi distinti, riflettendo la complessità delle rotte del mondo reale.
	\item \textbf{TSP Euclideo:} Questa variante posiziona le città in un piano euclideo, enfatizzando le distanze dirette e in linea retta, fondamentale per la navigazione dei droni e la pianificazione delle infrastrutture.
	\item \textbf{TSP con Finestre Temporali:} Aggiunge specifici intervalli di tempo entro i quali ogni città deve essere visitata, cruciale per la pianificazione nei servizi di consegna e nelle operazioni di manutenzione.
	\item \textbf{Problema di Routing dei Veicoli (VRP):} Un'estensione del \gls{TSP} che coinvolge più veicoli, centrale per ottimizzare la logistica della flotta e le reti di distribuzione.
\end{itemize}

L'applicazione del \gls{TSP} e delle sue varianti si estende a vari campi, mostrando la sua utilità nell'affrontare complessi problemi di ottimizzazione:

\begin{itemize}
	\item \textbf{Ottimizzazione Logistica:} Il \gls{TSP} euclideo e il VRP sono essenziali per snellire le rotte di consegna e migliorare l'efficienza logistica.
	\item \textbf{Produzione:} Il \gls{TSP} aiuta a ottimizzare i processi produttivi, riducendo i tempi di fermo e migliorando i flussi di lavoro complessivi.
	\item \textbf{Pianificazione e Scheduling:} Il \gls{TSP} con Finestre Temporali garantisce operazioni tempestive, ottimizzando l'allocazione delle risorse per i servizi con requisiti di tempistica critici.
	\item \textbf{Progettazione di Reti:} Gli algoritmi \gls{TSP} facilitano lo sviluppo di reti di telecomunicazioni e trasporti economiche ed efficienti.
	\item \textbf{Genomica:} In genetica, il \gls{TSP} supporta i progressi nel sequenziamento del genoma, offrendo informazioni sulle strutture e sui processi biologici complessi.
\end{itemize}

% \begin{tikzpicture}
%   \path[mindmap,concept color=orange!40,fill=orange!20] 
%     node[concept] {TSP Variants and Applications}
%     [clockwise from=0]
%     child[grow=0,concept color=blue!50,fill=blue!20] 
%       node[concept] {Symmetric TSP}
%       child[grow=0,concept color=purple!50,fill=purple!20] 
%         node[concept] {Logistics}
%       child[grow=90,concept color=green!50,fill=green!20] 
%         node[concept] {Manufacturing}
%     child[grow=60,concept color=red!50,fill=red!20] 
%       node[concept] {Asymmetric TSP}
%       child[grow=0,concept color=cyan!50,fill=cyan!20] 
%         node[concept] {Telecommunications}
%       child[grow=90,concept color=yellow!50,fill=yellow!20] 
%         node[concept] {Network Design}
%     child[grow=120,concept color=orange!50,fill=orange!20] 
%       node[concept] {Euclidean TSP}
%       child[grow=0,concept color=gray!50,fill=gray!20] 
%         node[concept] {Drone Navigation}
%       child[grow=90,concept color=blue!50,fill=blue!20] 
%         node[concept] {Infrastructure Planning}
%     child[grow=180,concept color=green!50,fill=green!20] 
%       node[concept] {TSP with Time Windows}
%       child[grow=0,concept color=red!50,fill=red!20] 
%         node[concept] {Delivery Services}
%       child[grow=90,concept color=purple!50,fill=purple!20] 
%         node[concept] {Maintenance Scheduling};
% \end{tikzpicture}

\section{Obiettivi della Tesi}



Questa tesi mira ad esplorare il Problema del Viaggiatore (TSP) sia da un punto di vista teorico che pratico. L'attenzione è rivolta all'esame del Sistema di Colonie di Formiche Red-Black, una variante dell'algoritmo di Ottimizzazione della Colonia di Formiche, con alcune modifiche, per comprenderne l'efficacia nell'affrontare le sfide del \gls{TSP}.  Lo studio ha due obiettivi: migliorare la comprensione del \gls{TSP} nell'ottimizzazione computazionale e valutare le prestazioni del Sistema di Colonie di Formiche Red-Black modificato rispetto alle soluzioni convenzionali.

\subsection{Obiettivi Principali}

La tesi è guidata dai seguenti obiettivi:

\begin{itemize}
	\item Esaminare il corpus esistente di lavori sul \gls{TSP}, raccogliendo le sue basi matematiche, l'evoluzione e l'insieme di strategie precedentemente ideate per la sua risoluzione.
	\item Dettagliare il funzionamento del Sistema di Colonie di Formiche Red-Black, enfatizzandone le basi teoriche, la meccanica operativa e le sue distinzioni rispetto agli approcci più tradizionali.
	\item Effettuare un confronto sperimentale sistematico del Sistema di Colonie di Formiche Red-Black contro gli algoritmi standard di risoluzione del \gls{TSP} su una gamma di insiemi di problemi, mirando a discernerne l'efficacia operativa e la scalabilità.
	\item Identificare gli scenari pratici in cui l'algoritmo modificato potrebbe essere applicato, mostrando la sua rilevanza e i potenziali benefici in contesti tangibili.
\end{itemize}

% TODO: this is nice, but needs fixing
% \begin{figure}[h!]
% 	\centering
% 	\begin{tikzpicture}[node distance=2cm, every node/.style={font=\small}]

% 		% Define the styles for the nodes
% 		\tikzstyle{startstop} = [rectangle, rounded corners, minimum width=3cm, minimum height=1cm,text centered, draw=black, fill=red!30]
% 		\tikzstyle{process} = [rectangle, minimum width=3cm, minimum height=1cm, text centered, draw=black, fill=orange!30]
% 		\tikzstyle{decision} = [diamond, minimum width=3cm, minimum height=1cm, text centered, draw=black, fill=green!30]
% 		\tikzstyle{arrow} = [thick,->,>=stealth]

% 		% Nodes
% 		\node (start) [startstop] {Start};
% 		\node (exact) [process, below of=start] {Exact Methods};
% 		\node (heuristic) [process, right of=exact, xshift=4cm] {Heuristic Methods};
% 		\node (branchbound) [process, below of=exact] {Branch and Bound};
% 		\node (ip) [process, below of=branchbound] {Integer Programming};
% 		\node (sa) [process, below of=heuristic] {Simulated Annealing};
% 		\node (ga) [process, below of=sa] {Genetic Algorithms};
% 		\node (end) [startstop, below of=ip] {End};

% 		% Arrows
% 		\draw [arrow] (start) -- (exact);
% 		\draw [arrow] (start) -- (heuristic);
% 		\draw [arrow] (exact) -- (branchbound);
% 		\draw [arrow] (exact) -- (ip);
% 		\draw [arrow] (heuristic) -- (sa);
% 		\draw [arrow] (heuristic) -- (ga);
% 		\draw [arrow] (branchbound) -- (end);
% 		\draw [arrow] (ip) -- (end);
% 		\draw [arrow] (sa) -- (end);
% 		\draw [arrow] (ga) -- (end);

% 	\end{tikzpicture}
% 	\caption{Flowchart of Exact and Heuristic Methods for Solving TSP}
% 	\label{fig:tsp_flowchart}
% \end{figure}
\subsection{Ambito dell'Indagine}

La ricerca approfondisce aree di interesse specifiche:

\begin{itemize}
	\item Una panoramica analitica del \gls{TSP}, che copre le sue definizioni, implicazioni e varianti, gettando le basi per le successive indagini algoritmiche.
	\item Una revisione approfondita degli algoritmi euristici e meteuristici, concentrandosi in particolare sull'Ottimizzazione della Colonia di Formiche come fondamento per il Sistema di Colonie di Formiche Red-Black.
	\item L'adattamento e l'applicazione del Sistema di Colonie di Formiche Red-Black, illustrando gli aspetti innovativi introdotti e le loro giustificazioni teoriche.
	\item Un confronto complessivo delle prestazioni degli algoritmi, utilizzando diverse istanze di \gls{TSP} per valutare l'efficacia e la qualità delle soluzioni.
\end{itemize}

Attraverso questa indagine, la tesi si sforza di offrire contributi significativi allo studio del \gls{TSP}, dimostrando l'utilità del Sistema di Colonie di Formiche Red-Black nella risoluzione di problemi di ottimizzazione e aprendo la strada a future ricerche.

\section{Ambito e Limitazioni}

Questa tesi di laurea si muove attraverso specifiche definizioni di ambito e incontra limitazioni cruciali da delineare per un'interpretazione accurata dei risultati dello studio e della loro rilevanza per il campo più ampio.

\subsection{Vincoli Teorici}

L'esplorazione all'interno di questa tesi è adattata per adattarsi alle risorse computazionali disponibili, concentrandosi su istanze di \gls{TSP} che bilanciano la presentazione di una sfida notevole e la fattibilità dell'analisi entro i vincoli della nostra configurazione computazionale. Un aspetto notevole di questa ricerca riguarda la reimplementazione in Rust di tutti gli algoritmi discussi, a causa della non disponibilità del codice sorgente originale per alcuni di questi algoritmi. Questa necessità mette in luce una possibile variazione nelle metriche di prestazione, che potrebbero non allinearsi completamente con quelle derivate dalle implementazioni originali degli autori. Tali differenze potrebbero derivare da vari fattori, tra cui le caratteristiche intrinseche di Rust come linguaggio di programmazione e le complessità coinvolte nel tradurre la logica algoritemica complessa in codice. Di conseguenza, mentre si mira a fornire un'analisi comparativa delle prestazioni algoritmiche, è imperativo considerare questi fattori di reimplementazione come potenziali variabili che influenzano i risultati.

Inoltre, mentre si riconosce il potenziale di applicazioni del mondo reale degli algoritmi in studio, gli studi approfonditi di casi che vadano oltre l'esplorazione teorica e la sperimentazione computazionale esulano dall'ambito di questa tesi di laurea. L'attenzione principale rimane sulle basi teoriche e sulla valutazione computazionale delle soluzioni \gls{TSP}.

\subsection{Considerazioni Pratiche}

La praticità di questa tesi è influenzata da diverse limitazioni che guidano il suo approccio sperimentale e la metodologia di valutazione:

\begin{itemize}
	\item La valutazione delle prestazioni del Sistema di Colonie di Formiche Red-Black viene condotta attraverso un sottoinsieme attentamente scelto di istanze di \gls{TSP}.  Queste istanze mirano a fornire una rappresentazione diversificata ma non esaustiva di potenziali scenari problematici, evidenziando il limite intrinseco nella cattura dell'intero spettro delle configurazioni di \gls{TSP}.
	\item A causa di limiti di risorse di calcolo, la scalabilità dell'algoritmo proposto viene testata in un ambiente vincolato. Questa limitazione potrebbe ostacolare l'estrapolazione dei risultati delle prestazioni a istanze di \gls{TSP} significativamente più grandi o complesse.
	\item La tesi considera le implicazioni pratiche delle sue scoperte in un contesto teorico e computazionale, senza addentrarsi in test sul campo o in studi di casi specifici per applicazioni.
\end{itemize}

Queste considerazioni sono fondamentali per modellare l'approccio della ricerca e interpretare i risultati di questa tesi di laurea, stabilendo aspettative realistiche per i suoi contributi alla comprensione e alla risoluzione del Problema del Viaggiatore.

\section{Struttura della Tesi}

\subsection{Panoramica dei Capitoli}
Il Capitolo \ref{chapt:1}
introduce il Problema del Viaggiatore (TSP), coprendo cosa sia, perché sia importante e un po' della sua storia. Esso delinea le principali domande a cui questa tesi mira a rispondere e delinea brevemente i metodi utilizzati.
Il Capitolo \ref{chapt:2}
analizza il quadro generale della complessità computazionale, concentrandosi sulla famosa questione P vs. NP, per aiutare a capire perché risolvere il \gls{TSP} può essere così impegnativo.
La discussione prosegue nei Capitoli \ref{chapt:3} e \ref{chapt:4} per esplorare i diversi modi di affrontare il \gls{TSP}.  Il Capitolo \ref{chapt:3} esamina le soluzioni esatte mentre il Capitolo \ref{chapt:4} discute i metodi euristici e meteuristici, incluse le ben note strategie come gli algoritmi genetici e il simulated annealing.
Il Capitolo \ref{chapt:5} si concentra sul Sistema di Colonie di Formiche, spiegando come funziona, perché è ispirato dalla natura e come si applica alla risoluzione del \gls{TSP}.
Quindi, il Capitolo \ref{chapt:6} introduce una variante del Sistema di Colonie di Formiche chiamata Sistema di Colonie di Formiche Red-Black. Questo capitolo spiega come è stato sviluppato, le nuove idee che porta e come i test dimostrano che potrebbe offrire un vantaggio nella risoluzione delle istanze di \gls{TSP}.
% Il Capitolo~\ref{chapt:7} presenta "Ibn-Battuta", uno strumento creato per visualizzare come diversi algoritmi \gls{TSP}, incluso il Sistema di Colonie di Formiche Red-Black, risolvono i problemi. Questo rende più facile vedere come funzionano gli algoritmi nella pratica.
Infine, il Capitolo \ref{chapt:7} conclude la tesi riassumendo i punti salienti di questa tesi.
Questa tesi mira a contribuire alla conversazione sulla risoluzione del \gls{TSP}, cercando di colmare il divario tra le sfide teoriche e le soluzioni del mondo reale, e aprire la porta a ulteriori indagini sui problemi di ottimizzazione.
