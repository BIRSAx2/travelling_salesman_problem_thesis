%!TEX root = ../template.tex
%%%%%%%%%%%%%%%%%%%%%%%%%%%%%%%%%%%%%%%%%%%%%%%%%%%%%%%%%%%%%%%%%%%%
%% chapter7.tex
%% NOVA thesis document file
%%
%% Chapter with lots of dummy text
%%%%%%%%%%%%%%%%%%%%%%%%%%%%%%%%%%%%%%%%%%%%%%%%%%%%%%%%%%%%%%%%%%%%

\typeout{NT FILE chapter7.tex}%

\chapter{Red-Black Ant Colony System}
\label{chapt:6}

\section{Introduzione}
Risolvere in modo efficiente istanze di grandi dimensioni del \gls{TSP} è un compito impegnativo nell'ambito dell'informatica. Il \gls{TSP} consiste nel trovare il tour più breve che visiti ciascuna città di un insieme dato esattamente una volta e ritorni alla città di partenza. Questo problema è un membro prominente della classe dei problemi di ottimizzazione combinatoria, il che lo rende computazionalmente difficile da risolvere in modo ottimale, soprattutto all'aumentare del numero di città.

L'algoritmo Ant Colony System (\gls{ACS}) si è affermato come un approccio promettente per risolvere istanze del \gls{TSP}. L'\gls{ACS} è ispirato dal comportamento di foraggiamento delle formiche reali, le quali utilizzano tracce di feromoni per comunicare e trovare percorsi efficienti tra il loro nido e le fonti di cibo. Simulando questa intelligenza collettiva, l'\gls{ACS} è in grado di convergere verso soluzioni di alta qualità per il \gls{TSP}. Tuttavia, quando la dimensione del problema cresce, l'algoritmo  \gls{ACS} standard può diventare troppo lento e fatica a trovare soluzioni ottimali.

Questo capitolo presenta una versione modificata dell'algoritmo \gls{ACS}, chiamata Red-Black Ant Colony System (\gls{RB-ACS}), che mira a migliorare le prestazioni dell'\gls{ACS} su istanze di grandi dimensioni del \gls{TSP}. Il \gls{RB-ACS} incorpora diverse migliorie chiave all'approccio  \gls{ACS} standard, tra cui la ricerca in parallelo, l'inizializzazione migliorata dei feromoni e l'impostazione di parametri differenziati per i due gruppi di formiche. Queste modifiche sono progettate per consentire al \gls{RB-ACS} di esplorare lo spazio di ricerca in modo più efficace e di convergere efficientemente verso soluzioni ottimali o quasi-ottimali, anche per problemi di \gls{TSP} su larga scala \cite{Hassan2013}.

\section{L'Algoritmo Red-Black Ant Colony System (\gls{RB-ACS})}
Mentre l'algoritmo  \gls{ACS} standard si è dimostrato efficace nella risoluzione di istanze del \gls{TSP}, le sue prestazioni possono peggiorare all'aumentare delle dimensioni del problema. Gli autori del presente lavoro propongono l'algoritmo Red-Black Ant Colony System (\gls{RB-ACS}), che introduce diverse modifiche chiave all'approccio  \gls{ACS} standard per migliorarne la scalabilità e la qualità delle soluzioni per problemi di \gls{TSP} su larga scala \cite{Hassan2013}.

\subsection{Inizializzazione dei Feromoni}
Nell' \gls{ACS} standard, il livello iniziale di feromone su tutti i bordi è impostato a un valore costante $\tau_0$. Nel \gls{RB-ACS}, gli autori propongono uno schema di inizializzazione dei feromoni diverso, in cui il livello iniziale di feromone su ciascun bordo $(r,s)$ è impostato secondo la seguente equazione:
\begin{equation}
	\tau_\text{init}(r,s) = \frac{C}{\text{costo}(r,s)}
\end{equation}
dove $C$ è una costante e $\text{costo}(r,s)$ è la distanza tra le città $r$ e $s$. Questa inizializzazione modificata dei feromoni incoraggia le formiche a esplorare i bordi più corti, che sono più desiderabili per il tour ottimale, assegnando loro livelli di feromone iniziali più alti \cite{Hassan2013}.

\subsection{Ricerca in Parallelo con Gruppi di Formiche Rosse e Nere}
Un'altra modifica chiave nell'algoritmo \gls{RB-ACS} è l'utilizzo di due gruppi separati di formiche, chiamati "rosse" e "nere", che esplorano lo spazio di ricerca in parallelo. Nell'\gls{ACS} standard, un singolo gruppo di formiche costruisce i tour e le formiche possono seguire i percorsi di altre formiche, il che può portare a una convergenza prematura e a rimanere bloccati in minimi locali.

Nel \gls{RB-ACS}, i gruppi di formiche rosse e nere operano in modo indipendente, mantenendo ciascuno i propri sentieri di feromoni e cercando soluzioni senza essere influenzati dall'altro gruppo. Questo approccio di ricerca parallela riduce la probabilità che l'algoritmo rimanga bloccato in minimi locali, poiché i due gruppi di formiche possono esplorare regioni diverse dello spazio di ricerca simultaneamente \cite{Hassan2013}.

\subsection{Impostazioni di Parametri Differenziate}
Oltre al processo di ricerca separato, l'algoritmo \gls{RB-ACS} impiega anche diverse impostazioni dei parametri per i gruppi di formiche rosse e nere. In particolare, la regola di aggiornamento locale dei feromoni e il tasso di evaporazione dei feromoni possono essere impostati su valori diversi per i due gruppi. Questa differenziazione è ispirata al comportamento osservato delle formiche reali, in cui colonie o gruppi diversi possono presentare caratteristiche distinte, come i tassi di deposizione e di evaporazione dei feromoni \cite{Hassan2013}.

Utilizzando impostazioni di parametri separate per le formiche rosse e nere, l'algoritmo \gls{RB-ACS} può ulteriormente migliorare la diversificazione del processo di ricerca, permettendo ai due gruppi di esplorare lo spazio di soluzione in modo più complementare.

\subsection{Aggiornamento Globale dei Feromoni Migliorato}
Nell' \gls{ACS} standard, solo la formica globalmente migliore è autorizzata a depositare feromoni durante la fase di aggiornamento globale. Nel \gls{RB-ACS}, gli autori propongono una regola di aggiornamento globale modificata, in cui le due migliori formiche di ciascuno dei gruppi rosso e nero sono autorizzate ad aggiornare i livelli di feromone. Questo aggiornamento globale parallelo rafforza ulteriormente la ricerca verso soluzioni di alta qualità, poiché vengono rinforzati simultaneamente diversi percorsi promettenti \cite{Hassan2013}.

\subsection{Pseudocodice dell'Algoritmo \gls{RB-ACS}}

% TODO: Inserire il pseudocodice dell'algoritmo \gls{RB-ACS}
\begin{algorithm}
	\caption{Red-Black Ant Colony System (\gls{RB-ACS}) per il \gls{TSP}}
	\begin{algorithmic}[1]
		\State Inizializza i livelli di feromoni $\tau_{ij} = \tau_{\text{init}}(i,j)$ per tutti gli archi $(i,j)$
		\State Inizializza la migliore soluzione globale $S_{gb}$ e la sua lunghezza $L_{gb}$
		\For{ogni iterazione}
		\For{ogni gruppo di formiche $g \in \{\text{rosso}, \text{nero}\}$}
		\For{ogni formica $k = 1, \ldots, m_g$}
		\State Posiziona la formica $k$ del gruppo $g$ su una città di partenza casuale
		\State Inizializza il tour parziale $S_k^g = \emptyset$
		\Repeat
		\State Seleziona la prossima città $j$ usando la regola di transizione specifica per il gruppo $g$
		\State Aggiungi $(i,j)$ a $S_k^g$
		\State Applica l'aggiornamento locale dei feromoni a $(i,j)$ secondo le regole del gruppo $g$
		\State $i \gets j$
		\Until{il tour $S_k^g$ è completo}
		\If{$L(S_k^g) < L_{gb}$}
		\State $S_{gb} \gets S_k^g$
		\State $L_{gb} \gets L(S_k^g)$
		\EndIf
		\EndFor
		\EndFor
		\For{ogni gruppo $g \in \{\text{rosso}, \text{nero}\}$}
		\State Seleziona le due migliori formiche del gruppo $g$
		\State Applica l'aggiornamento globale dei feromoni ai tour di queste formiche
		\EndFor
		\EndFor
		\State \Return la migliore soluzione trovata $S_{gb}$
	\end{algorithmic}
\end{algorithm}



\section{Risultati Sperimentali e Analisi}
Gli autori dell'algoritmo \gls{RB-ACS} ne hanno valutato le prestazioni su diverse istanze benchmark del \gls{TSP} tratte dalla libreria \gls{TSP}LIB \cite{TSPLIB}, tra cui i problemi Eil51, Eil76 e Kroa100. I risultati del \gls{RB-ACS} sono confrontati con l'algoritmo  \gls{ACS} standard, nonché con l'approccio Ant Colony Optimization with Multiple Ant Clans (\gls{ACOMAC}) \cite{Tsai2002}, utilizzando sia l'euristica del Vicino Più Vicino (Nearest Neighbor, NN) che quella del Doppio Vicino Più Vicino (Dual Nearest Neighbor, \gls{DNN}).

I risultati sperimentali mostrano che l'algoritmo proposto \gls{RB-ACS} supera gli altri approcci su tutte le istanze del \gls{TSP} testate. Per il problema Eil51 con 51 città, il \gls{RB-ACS} è in grado di trovare costantemente la soluzione ottimale di 426, con una lunghezza media del tour di 427,5, che è significativamente migliore dei risultati ottenuti da  \gls{ACS},  \gls{ACS}+\gls{DNN}, \gls{ACOMAC} e \gls{ACOMAC}+\gls{DNN}.

Miglioramenti simili sono osservati per i problemi Eil76 e Kroa100. Per il problema Eil76 con 76 città, la lunghezza media del tour prodotta dal \gls{RB-ACS} è 549,333, che è migliore degli altri algoritmi. Per il più grande problema Kroa100 con 100 città, il \gls{RB-ACS} raggiunge una lunghezza media del tour di 21389,235, superando gli altri metodi in modo considerevole.

Gli autori attribuiscono le prestazioni superiori del \gls{RB-ACS} ai vari miglioramenti introdotti, come l'inizializzazione migliorata dei feromoni, la ricerca parallela con gruppi di formiche rosse e nere, le impostazioni dei parametri differenziate e la regola di aggiornamento globale dei feromoni migliorata. Queste modifiche consentono al \gls{RB-ACS} di esplorare più efficacemente lo spazio di ricerca, evitare la convergenza prematura e convergere verso soluzioni ottimali o quasi-ottimali, anche per istanze di \gls{TSP} su larga scala \cite{Hassan2013}.

\section{Conclusione}
Questo capitolo ha presentato l'algoritmo \gls{RB-ACS}, una versione modificata dell'algoritmo standard \gls{ACS} per risolvere istanze di grandi dimensioni del  \gls{TSP} (TSP). Il \gls{RB-ACS} incorpora diverse migliorie chiave, tra cui l'inizializzazione migliorata dei feromoni, la ricerca parallela con gruppi di formiche rosse e nere, le impostazioni dei parametri differenziate e una regola di aggiornamento globale dei feromoni migliorata.

I risultati sperimentali su problemi benchmark del \gls{TSP} dimostrano che l'algoritmo \gls{RB-ACS} supera l' \gls{ACS} standard, nonché l'approccio Ant Colony Optimization with Multiple Ant Clans (\gls{ACOMAC}), in termini di qualità delle soluzioni e velocità di convergenza. Gli autori attribuiscono queste prestazioni migliorate alla capacità del \gls{RB-ACS} di esplorare più efficacemente lo spazio di ricerca e di evitare la convergenza prematura, soprattutto per istanze di \gls{TSP} di grandi dimensioni \cite{Hassan2013}.

L'algoritmo \gls{RB-ACS} presentato in questo capitolo può essere considerato un approccio promettente per risolvere problemi di ottimizzazione combinatoria complessi oltre il  \gls{TSP}, come il bilanciamento del carico nelle reti di telecomunicazioni, il dispacciamento economico del carico, la schedulazione dei processi e vari altri campi. I principi generali del \gls{RB-ACS}, tra cui la ricerca parallela, le impostazioni dei parametri differenziate e le strategie di aggiornamento migliorato dei feromoni, possono essere potenzialmente applicati a una vasta gamma di problemi di ottimizzazione, rendendolo un contributo prezioso nel campo dell'intelligenza di sciame e degli algoritmi metaeuristici.

