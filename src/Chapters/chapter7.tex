%!TEX root = ../template.tex
%%%%%%%%%%%%%%%%%%%%%%%%%%%%%%%%%%%%%%%%%%%%%%%%%%%%%%%%%%%%%%%%%%%%
%% chapter7.tex
%% NOVA thesis document file
%%
%% Chapter with lots of dummy text
%%%%%%%%%%%%%%%%%%%%%%%%%%%%%%%%%%%%%%%%%%%%%%%%%%%%%%%%%%%%%%%%%%%%

\typeout{NT FILE chapter7.tex}%

%!TEX root = ../dissertation.tex
\chapter{Conclusioni}
\label{chapt:7}

Questa tesi ha fornito una panoramica approfondita del Problema del Commesso Viaggiatore (TSP), affrontando sia gli aspetti teorici che gli approcci algoritmici per risolverlo.

Sono stati esaminati in dettaglio gli algoritmi esatti, come il metodo a forza bruta e l'algoritmo di Bellman-Held-Karp, analizzandone la fattibilità computazionale e i relativi vantaggi e svantaggi. Nonostante la loro capacità di trovare soluzioni ottimali, tali algoritmi risultano limitati nella pratica a istanze di dimensioni ridotte a causa della loro complessità computazionale.

Per affrontare istanze di dimensioni maggiori, sono stati approfonditi gli approcci euristici e metaeuristici, come gli algoritmi greedy, i metodi di local search (2-opt, 3-opt, Lin-Kernighan) e le metaeuristiche come Simulated Annealing, Algoritmi Genetici, Tabu Search e Ant Colony Optimization (ACO). Queste tecniche, pur non garantendo l'ottimalità, sono in grado di fornire soluzioni di buona qualità in tempi ragionevoli.

In particolare, è stato analizzato in dettaglio l'algoritmo Ant Colony System (ACS), una variante dell'ACO, evidenziandone i principi fondamentali, i dettagli di implementazione, la valutazione delle prestazioni e le principali varianti e estensioni.

Infine, è stato presentato l'algoritmo Red-Black Ant Colony System (RB-ACS), una proposta innovativa che introduce miglioramenti all'ACS attraverso l'utilizzo di gruppi di formiche rosse e nere che esplorano in parallelo lo spazio delle soluzioni, con impostazioni di parametri differenziate e un aggiornamento globale dei feromoni migliorato. I risultati sperimentali hanno dimostrato l'efficacia di questo approccio nel migliorare le prestazioni rispetto all'ACS originale.

In conclusione, questa tesi ha fornito una panoramica esaustiva delle principali tecniche per affrontare il TSP, evidenziando i punti di forza e di debolezza di ciascun approccio. L'approfondimento dell'algoritmo RB-ACS ha inoltre dimostrato come l'innovazione e la combinazione di diverse strategie possano portare a miglioramenti significativi nell'ambito della risoluzione del Problema del Commesso Viaggiatore.
