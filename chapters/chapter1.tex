% !TEX root = ../dissertation.tex
\chapter{Introduction}\label{chapt:1}
\label{introduction}

\section{Background Information}

\subsection{Definition and Importance of TSP}

The importance of TSP transcends its deceptively simple formulation, branching into various fields and applications. In logistics and transportation, solutions to TSP enable optimal routing that can lead to significant cost savings and efficiency improvements. In manufacturing, TSP algorithms are used in optimizing the movements of drills in PCB production or robotic arms in assembly lines to minimize production time. Beyond these, the TSP framework is employed in DNA sequencing, astronomy, and even in the creation of art, showcasing its versatility and significance in solving real-world problems.


\subsection{Historical Overview}

The Travelling Salesman Problem (TSP) represents one of the most studied challenges in computational mathematics and optimization, with a history that intertwines with the development of mathematical and operational research. Its evolution from early conceptual explorations to a pivotal problem in algorithm design is a testament to its complexity and widespread applicability.

The study of TSP can be traced back to the works of Sir William Rowan Hamilton and Thomas Penyngton Kirkman in the 19th century, focusing on Hamiltonian cycles and paths. Karl Menger, in the 1920s, introduced the "Messenger Problem," laying the groundwork for the modern formulation of TSP and emphasizing the quest for efficient routing.

The advent of computers and operational research techniques marked a significant phase in TSP research. Early pioneers like George Dantzig and Delbert Ray Fulkerson applied linear programming and cutting plane methods, illustrating the problem's computational complexity and its potential for practical applications.

By the 1940s, the TSP began to find relevance in agriculture and statistics, with researchers applying the problem's framework to optimize surveying and data collection processes, showcasing the TSP's versatility beyond theoretical mathematics.

In recent decades, the exploration of heuristic and bio-inspired algorithms, notably Marco Dorigo's work on ant colony optimization, has opened new avenues for solving TSP, highlighting the continued innovation in tackling its NP-hardness.

\section{Problem Statement}

\textbf{The Travelling Salesman Problem:} Given a set of cities and the distances between each pair of cities, find the shortest possible route that visits each city exactly once and returns to the original city.

\subsection{Mathematical Formulation}

Given a graph \(G = (V, E)\), where \(V\) is the set of vertices (cities) and \(E\) is the set of edges (paths between cities), with each edge \((i, j) \in E\) assigned a weight \(w_{ij}\) representing the travel cost from city \(i\) to city \(j\), the TSP seeks a Hamiltonian cycle of minimal weight.
The objective function to minimize is:
\[
	\text{Minimize} \sum_{i=1}^{n} \sum_{j \ne i, j=1}^{n} w_{ij} x_{ij},
\]
subject to the constraints:
\[
	\sum_{i=1, i \ne j}^{n} x_{ij} = 1, \quad \forall j \in V,
\]
\[
	\sum_{j=1, j \ne i}^{n} x_{ij} = 1, \quad \forall i \in V,
\]
alongside subtour elimination constraints, ensuring each city is visited exactly once and the tour is cyclic.

\subsection{Integer Linear Programming Formulations}

\subsubsection{Miller–Tucker–Zemlin Formulation}

The MTZ formulation introduces auxiliary variables \(u_i\) for ordering, alongside the binary variables \(x_{ij}\):

Objective:
\[
	\min \sum_{i=1}^{n} \sum_{j \ne i, j=1}^{n} c_{ij} x_{ij},
\]
Subject to:
\[
	\sum_{i=1, i \ne j}^{n} x_{ij} = 1, \quad \forall j,
\]
\[
	\sum_{j=1, j \ne i}^{n} x_{ij} = 1, \quad \forall i,
\]
\[
	u_i - u_j + 1 \le (n-1)(1 - x_{ij}), \quad \forall 2 \le i \ne j \le n,
\]
\[
	2 \le u_i \le n, \quad \forall 2 \le i \le n.
\]

This formulation enforces a single tour through auxiliary ordering variables, preventing subtours effectively.

\subsubsection{Dantzig–Fulkerson–Johnson Formulation}

The DFJ formulation, known for its efficiency, introduces subtour elimination constraints directly:

Objective:
\[
	\min \sum_{i=1}^{n} \sum_{j \ne i, j=1}^{n} c_{ij} x_{ij},
\]
Subject to:
\[
	\sum_{i=1, i \ne j}^{n} x_{ij} = 1, \quad \forall j,
\]
\[
	\sum_{j=1, j \ne i}^{n} x_{ij} = 1, \quad \forall i,
\]
\[
	\sum_{i \in S, j \notin S} x_{ij} \ge 1, \quad \forall S \subset V, S \ne \emptyset, S \ne V.
\]

This formulation combats subtours by ensuring at least one edge leads out of every subset of vertices, guaranteeing a single, connected tour.

\subsection{Variants and Applications}

The TSP adapts to a broad array of practical challenges, demonstrating its versatility. Key variants include:

\begin{itemize}
	\item \textbf{Symmetric vs. Asymmetric TSP (STSP vs. ATSP):} STSP assumes identical travel costs between any two cities, whereas ATSP allows for distinct costs, reflecting the complexity of real-world routes.
	\item \textbf{Euclidean TSP:} This variant situates cities within a Euclidean plane, emphasizing direct, straight-line distances, pivotal for drone navigation and infrastructure planning.
	\item \textbf{Time Windows TSP:} It adds specific time frames within which each city must be visited, crucial for scheduling in delivery services and maintenance operations.
	\item \textbf{Vehicle Routing Problem (VRP):} An extension of TSP that involves multiple vehicles, central to optimizing fleet logistics and distribution networks.
\end{itemize}

The application of TSP and its variants extends across various fields, showcasing its utility in tackling complex optimization problems:

\begin{itemize}
	\item \textbf{Logistics Optimization:} Euclidean TSP and VRP are instrumental in streamlining delivery routes and enhancing logistics efficiency.
	\item \textbf{Manufacturing:} TSP aids in optimizing production processes, reducing downtime, and improving overall manufacturing workflows.
	\item \textbf{Scheduling and Planning:} Time Windows TSP ensures timely operations, optimizing resource allocation for services with critical timing requirements.
	\item \textbf{Network Design:} TSP algorithms facilitate the development of cost-effective and efficient telecommunications and transportation networks.
	\item \textbf{Genomics:} In genetics, TSP supports advancements in genome sequencing, offering insights into complex biological structures and processes.
\end{itemize}
\section{Objectives of the Thesis}

This thesis aims to explore the Travelling Salesman Problem (TSP) from both a theoretical and practical standpoint. The focus is on examining the Red-Black Ant Colony System—a variation of the Ant Colony Optimization algorithm—with some modifications, to understand its effectiveness in addressing TSP challenges. The study has dual objectives: enhancing the comprehension of TSP in computational optimization and evaluating the modified Red-Black Ant Colony System's performance against conventional solutions.

\subsection{Main Goals}

The research is guided by the following objectives:

\begin{itemize}
	\item To survey the existing body of work on TSP, encapsulating its mathematical underpinnings, evolution, and the array of strategies previously devised for its resolution.
	\item To detail the workings of the Red-Black Ant Colony System, emphasizing its theoretical basis, operational mechanics, and its distinctions from more traditional approaches.
	\item To perform a methodical experimental comparison of the Red-Black Ant Colony System against standard TSP-solving algorithms over a spectrum of problem sets, aiming to discern its operational efficacy and scalability.
	\item To identify practical scenarios where the modified algorithm could be applied, showcasing its relevance and potential benefits in tangible settings.
\end{itemize}

\subsection{Scope of Investigation}

The research delves into specific areas of interest:

\begin{itemize}
	\item An analytical overview of TSP, covering its definitions, implications, and variations, laying the groundwork for subsequent algorithmic investigations.
	\item A thorough review of heuristic and metaheuristic algorithms, focusing particularly on Ant Colony Optimization as the foundation for the Red-Black Ant Colony System.
	\item The adaptation and application of the Red-Black Ant Colony System, elucidating the novel aspects introduced and their theoretical justifications.
	\item A comprehensive comparison of algorithmic performances, using diverse TSP instances to evaluate effectiveness and solution quality.
\end{itemize}

Through this investigation, the thesis endeavors to offer meaningful contributions to the study of TSP, demonstrating the utility of the Red-Black Ant Colony System in solving optimization problems and paving the way for future inquiries.

\section{Scope and Limitations}

This bachelor thesis navigates through specific scope definitions and encounters limitations that are crucial to delineate for an accurate interpretation of the study's outcomes and their relevance to the broader field.

\subsection{Theoretical Constraints}

The exploration within this thesis is tailored to fit the computational resources at hand, focusing on TSP instances that balance between presenting a notable challenge and being feasible for analysis within the constraints of our computational setup. A notable aspect of this research involves the re-implementation of all algorithms discussed in Rust, owing to the unavailability of the original source code for some of these algorithms. This necessity brings to light a potential variation in the performance metrics, which may not fully align with those derived from the original authors' implementations. Such differences could arise from various factors, including the inherent characteristics of Rust as a programming language and the intricacies involved in translating complex algorithmic logic into code. Consequently, while aiming to provide a comparative analysis of algorithmic performance, it's imperative to consider these re-implementation factors as potential variables affecting the results.

Moreover, while there is an acknowledgment of the potential real-world applications of the algorithms under study, in-depth case studies extending beyond theoretical exploration and computational experimentation fall outside the purview of this bachelor thesis. The primary focus remains on the theoretical underpinnings and computational assessment of TSP solutions.

\subsection{Practical Considerations}

The practicality of this thesis is influenced by several limitations that guide its experimental approach and evaluation methodology:

\begin{itemize}
	\item The evaluation of the Red-Black Ant Colony System's performance is conducted through a carefully chosen subset of TSP instances. These instances aim to provide a diverse yet not exhaustive representation of potential problem scenarios, highlighting the inherent limitation in capturing the full spectrum of TSP configurations.
	\item Due to computational resource limitations, the scalability of the proposed algorithm is tested within a constrained environment. This limitation may impede the extrapolation of performance results to significantly larger or more complex TSP instances.
	\item The thesis considers the practical implications of its findings within a theoretical and computational context, without venturing into extensive field testing or application-specific case studies.
\end{itemize}

These considerations are fundamental in shaping the research approach and interpreting the findings of this bachelor thesis, setting realistic expectations for its contributions to the understanding and solving of the Travelling Salesman Problem.

\section{Thesis Structure}

\subsection{Overview of Chapters}

Chapter~\ref{chapt:1} introduces the Travelling Salesman Problem (TSP), covering what it is, why it matters, and a bit about its history. It sets up the main questions this thesis aims to answer and briefly outlines the methods used.

Chapter~\ref{chapt:2} looks at the big picture of computational complexity, focusing on the famous P vs. NP question, to help understand why solving the TSP can be so challenging.

The discussion moves on in Chapters \ref{chapt:3} and ~\ref{chapt:4} to explore different ways to tackle the TSP. Chapter \ref{chapt:3} looks at exact solutions while Chapter ~\ref{chapt:4} discusses heuristic and metaheuristic methods, including well-known strategies like genetic algorithms and simulated annealing.

Chapter~\ref{chapt:5} zooms in on the Ant Colony System, explaining how it works, why it's inspired by nature, and how it applies to solving the TSP.

Then, Chapter~\ref{chapt:6} introduces a twist on the Ant Colony System called the Red-Black Ant Colony System. This chapter explains how it was developed, the new ideas it brings, and how testing shows it might offer an edge in solving TSP instances.

Chapter~\ref{chapt:7} presents "Ibn-Battuta," a tool created to visualize how different TSP algorithms, including the Red-Black Ant Colony System, solve problems. This makes it easier to see how the algorithms work in practice.

Finally, Chapter~\ref{chapt:8} wraps up the thesis by summarizing the main discoveries, particularly how the Red-Black Ant Colony System performs. It recognizes the study's limitations and suggests directions for further research, pointing out how these findings could be built upon or applied in new ways.

This thesis aims to add to the conversation on solving the TSP, hoping to bridge the gap between theoretical challenges and real-world solutions, and open the door for more investigations into optimization problems.
