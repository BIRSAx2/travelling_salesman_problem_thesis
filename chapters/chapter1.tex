%!TEX root = ../dissertation.tex
\chapter{Introduction}\label{chapt:1}
\label{introduction}

\section{Background Information}

\subsection{Definition and Importance of TSP}
The Travelling Salesman Problem (TSP) is one of the most studied problems in computational mathematics and operations research. It can be succinctly described as a problem where a salesman seeks to find the shortest possible route that visits a set of cities exactly once and returns to the original city, with the aim of minimizing the total journey distance. Formally, TSP is defined on a graph where nodes represent cities, and edges represent the distances between them. The problem asks for the permutation of cities that results in the shortest possible tour.

The importance of TSP transcends its deceptively simple formulation, branching into various fields and applications. In logistics and transportation, solutions to TSP enable optimal routing that can lead to significant cost savings and efficiency improvements. In manufacturing, TSP algorithms are used in optimizing the movements of drills in PCB production or robotic arms in assembly lines to minimize production time. Beyond these, the TSP framework is employed in DNA sequencing, astronomy, and even in the creation of art, showcasing its versatility and significance in solving real-world problems.


\subsection{Historical Overview}

The origins of the TSP are not precisely documented, but its mathematical study can be traced back to the 18th and 19th centuries with the work of mathematicians such as Sir William Rowan Hamilton and Thomas Penyngton Kirkman. Hamilton and Kirkman explored problems related to Hamiltonian cycles and tours, which are closely related to what is known today as the TSP.

The explicit formulation of TSP as it is recognized today was introduced in the 1930s. However, it was the advent of computers and the development of operational research techniques in the mid-20th century that truly catalyzed the intensive study of TSP. Mathematicians and computer scientists, including George Dantzig, Delbert Ray Fulkerson, and Selmer M. Johnson, made significant early contributions by applying linear programming and cutting plane methods to solve instances of TSP, laying the groundwork for decades of research to follow.

In the subsequent years, the TSP has continued to captivate researchers due to its NP-hardness—a classification that implies there is no known algorithm that can solve all instances of the problem efficiently (in polynomial time). This inherent complexity, coupled with the TSP's applicability to a wide array of practical problems, has made it a centerpiece in the study of algorithm design and optimization.


\section{Problem Statement}

\subsection{Mathematical Formulation}


The Travelling Salesman Problem (TSP) is formally defined on a graph \(G = (V, E)\), where \(V\) represents the set of vertices (cities) and \(E\) the set of edges (paths between cities). Each edge \((i, j) \in E\) is associated with a weight \(w_{ij}\), which represents the cost (distance, time, or expense) of traveling from city \(i\) to city \(j\). The objective is to find a Hamiltonian cycle (a cycle that visits each vertex exactly once and returns to the starting vertex) with the minimum possible total weight.

Mathematically, the problem can be expressed as an optimization problem. Let \(x_{ij}\) be a binary variable that equals 1 if the tour moves directly from city \(i\) to city \(j\) and 0 otherwise. The goal is to minimize the total cost of the tour:

\[
    \text{Minimize} \sum_{i=1}^{n} \sum_{j \ne i, j=1}^{n} w_{ij} x_{ij}
\]

subject to:

\[
    \sum_{i=1, i \ne j}^{n} x_{ij} = 1, \quad \forall j \in V
\]

\[
    \sum_{j=1, j \ne i}^{n} x_{ij} = 1, \quad \forall i \in V
\]

\[
    \text{and the subtour elimination constraints.}
\]

The first two sets of constraints ensure that each city is entered and left exactly once. The subtour elimination constraints are more complex and are necessary to prevent the solution from containing separate cycles that do not constitute a valid tour.

\subsection{Variants and Applications}

% TODO Adjust spacing

The TSP exhibits numerous variants, broadening its application to a variety of real-world scenarios. Key variants include:

- \textbf{Symmetric vs. Asymmetric TSP (STSP vs. ATSP)}: In the STSP, the distance from city $i$ to city $j$ is identical to that from city $j$ to city $i$, while in the ATSP, these distances may differ, mirroring real scenarios such as one-way streets.

- \textbf{Euclidean TSP}: Cities are points within the Euclidean plane, and distances are Euclidean distances, relevant to logistics and drone navigation.

- \textbf{Time Windows TSP}: Each city must be visited within a specific timeframe, adding temporal dimensions to the routing challenge, applicable in delivery services and scheduling.

- \textbf{Vehicle Routing Problem (VRP)}: An extension of the TSP where multiple vehicles are available for deliveries or pickups, used in fleet logistics and distribution management.

The applications for TSP and its variants span across diverse fields including logistics optimization, manufacturing, scheduling, network design, and genetics for genome sequencing. This illustrates the TSP's universality not only as a theoretical challenge in computer science and operations research but also as a practical tool for addressing complex optimization issues across various industries and scientific domains.

\section{Objectives of the Thesis}

The overarching objective of this thesis is to delve into the intricacies of the Travelling Salesman Problem (TSP), with a particular focus on exploring innovative algorithmic solutions and their applications. This exploration is twofold: firstly, to extend the theoretical understanding of TSP within the realm of computational complexity and optimization; and secondly, to develop and assess the efficacy of a novel algorithmic approach, the Red-Black Ant Colony System, in solving instances of TSP more efficiently than existing methodologies.

\subsection{Main Goals}

The thesis is driven by several key goals:

\begin{itemize}
    \item To provide a comprehensive review of existing literature on TSP, including its mathematical formulation, historical development, and the various algorithmic strategies previously employed to tackle the problem.
    \item To introduce the Red-Black Ant Colony System, a novel variant of the Ant Colony Optimization technique, detailing its conceptual foundation, algorithmic structure, and potential advantages over traditional methods.
    \item To conduct a rigorous experimental analysis comparing the Red-Black Ant Colony System with established TSP algorithms across various problem instances, thereby evaluating its performance and scalability.
    \item To demonstrate the practical applications of the proposed algorithm in real-world scenarios, thereby illustrating its utility beyond theoretical interest.
\end{itemize}

\subsection{Scope of Investigation}

This thesis confines its investigation to the following areas:

\begin{itemize}
    \item A detailed examination of the TSP, including its definitions, significance, and variants, to set the stage for further algorithmic exploration.
    \item An in-depth analysis of heuristic and metaheuristic algorithms, with a spotlight on Ant Colony Optimization as a precursor to the proposed Red-Black Ant Colony System.
    \item The development and implementation of the Red-Black Ant Colony System, including its algorithmic innovations and operational principles.
    \item An extensive comparative study of algorithmic performance, utilizing a variety of TSP instances to assess both efficiency and solution quality.
    \item Exploration of the broader implications of the research findings, including potential applications in logistics, scheduling, and network design, and a discussion on future research directions.
\end{itemize}

The thesis aims to contribute valuable insights into the TSP domain, pushing the boundaries of existing algorithmic approaches and opening new avenues for practical applications and further research.


\section{Scope and Limitations}

This thesis is subject to certain scope definitions and limitations that delineate the extent of the investigation and the applicability of its findings. These limitations are grounded in both theoretical constraints and practical considerations, which are essential to acknowledge for a comprehensive understanding of the study's context and its contributions to the field.

\subsection{Theoretical Constraints}

Practically, the thesis is bounded by the computational resources available for the testing and evaluation of the proposed algorithms. The experimental analysis focuses on instances of the TSP that pose a significant challenge yet remain manageable within the constraints of the available computational framework. Furthermore, this study has necessitated the re-implementation of all discussed algorithms in Rust, a decision driven by the unavailability of original source code for some algorithms. It is important to note that this re-implementation could potentially impact the performance comparison between the newly implemented algorithms and the existing ones. The discrepancies in performance metrics could stem not solely from algorithmic efficiency but also from the intrinsic differences in programming languages and the possible nuances lost or gained in the translation of algorithmic logic to Rust. Additionally, while the thesis briefly explores the potential real-world applications of the developed system, conducting in-depth case studies falls outside its scope, prioritizing instead the theoretical and computational exploration of TSP solutions.

\subsection{Practical Considerations}

Practical limitations also shape the scope of this thesis, influencing the design, execution, and evaluation of the proposed solutions:


\begin{itemize}
    \item The performance assessment of the Red-Black Ant Colony System is based on a selected set of TSP instances, which, while diverse, cannot encompass the full spectrum of possible problem configurations.
    \item Computational resource constraints limit the scalability testing of the proposed algorithm, potentially affecting the generalization of performance results to larger problem instances.
    \item The practical applicability of findings is considered within the contexts provided


\end{itemize}
\section{Thesis Structure}

\subsection{Overview of Chapters}

Chapter~\ref{chapt:1} lays the foundation by delving into the TSP's definition, significance, and historical context, subsequently framing the problem statement, thesis objectives, and methodological approach. The discourse progresses to Chapter~\ref{chapt:2}, where the theoretical underpinnings of computational complexity, particularly the P vs. NP problem, are examined to contextualize the TSP's computational challenges.

In Chapters \ref{chapt:3} and ~\ref{chapt:4}, the thesis scrutinizes both exact and heuristic/metaheuristic algorithms, respectively, providing a critical analysis of methods ranging from the brute force approach to more nuanced strategies like genetic algorithms and simulated annealing. Special attention is devoted to the original Ant Colony System by Dorigo et al. in Chapter ~\ref{chapt:5}, dissecting its biological inspiration, algorithmic principles, and application to TSP.

The narrative then shifts in Chapter~\ref{chapt:6} to spotlight the Red-Black Ant Colony System, detailing its conceptualization, algorithmic innovations, and the experimental analysis underscoring its comparative advantage. Chapter~\ref{chapt:7} introduces "Ibn-Battuta," a bespoke software visualizer for TSP algorithms, designed to elucidate the operational dynamics of various algorithms, including the newly proposed system.

Concluding in Chapter~\ref{chapt:8}, the thesis synthesizes the findings, emphasizing the Red-Black Ant Colony System's performance and the broader theoretical and practical implications of the study. While acknowledging limitations, the thesis also proposes avenues for future research, suggesting potential improvements and broader applications of the findings.

This work not only contributes a novel perspective to the algorithmic tackling of TSP but also reinforces the interdisciplinary bridge between computational theory and practical application, paving the way for future explorations in optimization problems.