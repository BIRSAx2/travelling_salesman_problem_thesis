\chapter{Dual Ant Colony System}\label{chapt:6}

\section{Development}

\subsection{Conceptualization}
DACS enhances the traditional Ant Colony System by incorporating dual ant populations for solving large-scale Traveling Salesman Problems (TSP). It aims to address the scalability issues of existing algorithms, leveraging the concept of parallel search inspired by genetic algorithms to accelerate the convergence towards optimal or near-optimal solutions. The innovation lies in its ability to maintain diversity in the search process, thereby reducing the likelihood of premature convergence.

\subsection{Algorithmic Innovations}
The core innovation of DACS is the introduction of red and black ant populations that explore the solution space independently. This strategy is rooted in the observation that different ant colonies in nature often exhibit unique foraging behaviors, which can be analogously applied to optimize search strategies in computational algorithms. By separating the search processes, DACS effectively doubles the exploration capacity without significantly increasing computational complexity.


\section{Algorithm Design}

\subsection{Pseudo-code Presentation}
The following pseudocode provides a detailed overview of the DACS algorithm, highlighting the separate handling of red and black ant populations and their interaction through pheromone updates.

\begin{algorithm}
	\caption{Detailed Dual Ant Colony System (DACS) for TSP}
	\begin{algorithmic}[1]
		\State Initialize pheromones on all edges based on Eqn. (6), considering distance.
		\State Distribute ants of red and black populations randomly among cities.
		\Repeat
		\For{each ant group (red and black)}
		\For{each ant $k$ in the group}
		\State Construct a tour based on the state transition rule (Eqn. (1) and Eqn. (2)).
		\State Apply local updating rule (Eqn. (3)) to visited edges.
		\EndFor
		\EndFor
		\State Identify the best tours from each group.
		\State Apply the global updating rule (Eqn. (4)) based on the best tours.
		\Until{a stopping criterion is satisfied}
		\State Output the shortest tour found.
	\end{algorithmic}
\end{algorithm}

\subsection{Operational Principles}
DACS operationally diverges from traditional ACS by introducing cost-based pheromone initialization, which inversely relates edge cost to initial pheromone levels, encouraging exploration of less costly paths early in the search. Additionally, by employing two ant populations with distinct characteristics (pheromone sensitivity, evaporation rates), the system fosters a broader and more effective exploration and exploitation strategy, significantly enhancing performance on large TSP instances.

\section{Experimental Analysis}

\subsection{Setup and Methodology}
The DACS was rigorously tested against benchmark problems from the TSPLIB repository, involving symmetric and asymmetric TSP instances. The algorithm's parameters, including the pheromone decay rate ($\alpha$), relative importance of pheromone versus distance ($\beta$), and the trail persistence ($\rho$), were optimized through preliminary trials. Each problem instance was run 30 times to ensure statistical reliability of the results.

\subsection{Results and Comparisons}
The experimental results underscored the effectiveness of DACS, with significant improvements observed across various TSP instances compared to traditional ACS and other variations like ACS+DNN and ACOMAC+DNN. Specifically, DACS consistently found tours closer to the optimal solutions with fewer iterations, highlighting its efficiency and robustness for large-scale optimization challenges.

