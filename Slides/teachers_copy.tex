\documentclass{beamer}
\usepackage[utf8]{inputenc}
\usepackage[italian]{babel}
\usepackage{graphicx}
\usepackage{booktabs}
\usepackage{tikz}
\usepackage{adjustbox}

\usetheme{Amurmaple}
\setbeamertemplate{navigation symbols}{}%remove navigation symbols
\title[Travelling Salesman Problem]{Esplorazione delle Soluzioni al TSP: Algoritmi Esatti e Red-Black ACS}
\author{Mouhieddine Sabir}
\institute[UniPD]{Università degli Studi di Padova}
\date{Settembre 2024}

\begin{document}

\frame{\titlepage}

\begin{frame}
    \frametitle{Sommario}
    \tableofcontents
\end{frame}

\section{Introduzione}
\tiny
\begin{frame}{Introduzione al TSP e Formulazione Matematica}
    \begin{block}{Definizione del Problema del Commesso Viaggiatore (TSP)}
        \begin{itemize}
            \item Trovare il percorso più breve che visiti ogni città una volta e torni all'inizio
            \item Applicazioni: logistica, ingegneria, genomica, astronomia
        \end{itemize}
    \end{block}

    \begin{block}{Formulazione Matematica}
        \begin{itemize}
            \item Modello: minimizzare $\sum_{i=1}^{n} \sum_{j=1, j \neq i}^{n} c_{ij} x_{ij}$ su grafo $G = (V, E)$
            \item Vincoli: ogni città visitata una sola volta
            \item Varianti: STSP, ATSP, TSPTW
        \end{itemize}
    \end{block}

    \begin{block}{Algoritmi Esatti}
        \begin{itemize}
            \item Brute Force: $O(n!)$, per problemi piccoli
            \item Bellman-Held-Karp: $O(n^2 \cdot 2^n)$, programmazione dinamica
            \item Concorde TSP Solver: branch-and-cut, per istanze grandi
        \end{itemize}
    \end{block}
\end{frame}

\section{Algoritmi Euristici e Metaeuristici}
\begin{frame}{Algoritmi Euristici}
    \textbf{Nearest Neighbor (NNS)}
    \begin{itemize}
        \item Seleziona la città più vicina non ancora visitata.
        \item Soluzione veloce, ma non ottimale.
    \end{itemize}
    \textbf{2-opt e 3-opt}
    \begin{itemize}
        \item Tecniche di ottimizzazione locale: scambiano segmenti del tour per ridurre la lunghezza.
        \item 2-opt: scambio di due lati.
        \item 3-opt: scambio di tre lati.
    \end{itemize}


    \begin{table}
        \centering
        \caption{Risultati NN vs NN+2Opt}
        \begin{tabular}{lllrrrr}
            \toprule
               & Istanza  & algorithm & Tempo (ms) & Lunghezza Tour & Lunghezza ottima & Gap   \\
            \midrule
            0  & berlin52 & NN        & 14         & 8980.92        & 7542.00          & 19.08 \\
            1  & berlin52 & NN2Opt    & 124        & 8060.65        & 7542.00          & 6.88  \\
            2  & d198     & NN        & 183        & 18620.07       & 15780.00         & 18.00 \\
            3  & d198     & NN2Opt    & 4443       & 16165.31       & 15780.00         & 2.44  \\
            4  & eil76    & NN        & 17         & 711.99         & 538.00           & 32.34 \\
            5  & eil76    & NN2Opt    & 313        & 599.05         & 538.00           & 11.35 \\
            6  & fl1577   & NN        & 9547       & 27940.91       & 22249.00         & 25.58 \\
            7  & fl1577   & NN2Opt    & 200151     & 24214.30       & 22249.00         & 8.83  \\
            8  & lin105   & NN        & 43         & 20362.76       & 14379.00         & 41.61 \\
            9  & lin105   & NN2Opt    & 2959       & 16199.70       & 14379.00         & 12.66 \\
            10 & lin318   & NN        & 466        & 54033.58       & 42029.00         & 28.56 \\
            11 & lin318   & NN2Opt    & 9562       & 46408.41       & 42029.00         & 10.42 \\
            12 & rl5915   & NN        & 236473     & 707498.63      & 565530.00        & 25.10 \\
            13 & rl5915   & NN2Opt    & 6327036    & 620822.08      & 565530.00        & 9.78  \\
            14 & u574     & NN        & 1295       & 46881.87       & 36905.00         & 27.03 \\
            15 & u574     & NN2Opt    & 26923      & 39896.00       & 36905.00         & 8.10  \\
            \bottomrule
        \end{tabular}
    \end{table}
\end{frame}
\section{Algoritmi Metaeuristici}
\begin{frame}{Simulated Annealing (SA)}
    \textbf{Principio di Base}
    \begin{itemize}
        \item Ispirato al processo fisico della tempra dei metalli, dove una sostanza viene riscaldata e poi raffreddata lentamente per raggiungere uno stato a bassa energia.
        \item Simulated Annealing (SA) applica questo concetto all'ottimizzazione, cercando di evitare ottimi locali accettando temporaneamente soluzioni peggiori con una certa probabilità.
    \end{itemize}
    \begin{table}[H]
        \centering
        \caption{Risultati SA vs SA2Opt}
        \begin{tabular}{lllrrrr}
            \toprule
               & Istanza  & algorithm & Tempo (ms) & Lunghezza Tour & Lunghezza ottima & Gap   \\
            \midrule
            0  & berlin52 & SA        & 28982      & 7544.37        & 7542.00          & 0.03  \\
            1  & berlin52 & SA2Opt    & 49481      & 7544.37        & 7542.00          & 0.03  \\
            2  & d198     & SA2Opt    & 93813      & 16118.48       & 15780.00         & 2.14  \\
            3  & d198     & SA        & 102362     & 16318.76       & 15780.00         & 3.41  \\
            4  & eil76    & SA        & 39587      & 572.81         & 538.00           & 6.47  \\
            5  & eil76    & SA2Opt    & 45721      & 569.29         & 538.00           & 5.82  \\
            6  & fl1577   & SA        & 1514830    & 27584.16       & 22249.00         & 23.98 \\
            7  & fl1577   & SA2Opt    & 2391057    & 23489.49       & 22249.00         & 5.58  \\
            8  & lin105   & SA        & 52928      & 14993.92       & 14379.00         & 4.28  \\
            9  & lin105   & SA2Opt    & 54366      & 14882.69       & 14379.00         & 3.50  \\
            10 & lin318   & SA2Opt    & 392433     & 45444.25       & 42029.00         & 8.13  \\
            11 & lin318   & SA        & 506369     & 47651.44       & 42029.00         & 13.38 \\
            12 & rl5915   & SA2Opt    & 18690779   & 615257.00      & 565530.00        & 8.79  \\
            13 & rl5915   & SA        & 22563370   & 680777.61      & 565530.00        & 20.38 \\
            14 & u574     & SA2Opt    & 369840     & 39443.68       & 36905.00         & 6.88  \\
            15 & u574     & SA        & 398181     & 43936.09       & 36905.00         & 19.05 \\
            \bottomrule
        \end{tabular}
    \end{table}
\end{frame}

\begin{frame}{Algoritmi Genetici (GA)}
    \textbf{Principio di Base}
    \begin{itemize}
        \item Ispirato dalla teoria dell'evoluzione naturale di Darwin. Le soluzioni del problema sono trattate come individui in una popolazione che evolve nel tempo.
        \item Utilizza concetti di selezione, crossover (incrocio) e mutazione per generare nuove soluzioni.
    \end{itemize}
    \begin{table}[H]
        \centering

        \caption{Risultati GA vs GA+2Opt}
        \begin{tabular}{lllrrrr}
            \toprule
               & Istanza  & algorithm & Tempo (ms) & Lunghezza Tour & Lunghezza ottima & Gap     \\
            \midrule
            0  & berlin52 & GA2Opt    & 26104      & 7918.09        & 7542.00          & 4.99    \\
            1  & berlin52 & GA        & 29932      & 11009.32       & 7542.00          & 45.97   \\
            2  & d198     & GA2Opt    & 108498     & 16548.42       & 15780.00         & 4.87    \\
            3  & d198     & GA        & 137106     & 66868.76       & 15780.00         & 323.76  \\
            4  & eil76    & GA2Opt    & 44547      & 570.63         & 538.00           & 6.06    \\
            5  & eil76    & GA        & 45556      & 977.18         & 538.00           & 81.63   \\
            6  & fl1577   & GA        & 2264528    & 1116417.45     & 22249.00         & 4917.83 \\
            7  & fl1577   & GA2Opt    & 4033113    & 24024.81       & 22249.00         & 7.98    \\
            8  & lin105   & GA2Opt    & 109871     & 14573.34       & 14379.00         & 1.35    \\
            9  & lin105   & GA        & 164073     & 44653.50       & 14379.00         & 210.55  \\
            10 & lin318   & GA2Opt    & 320190     & 45165.59       & 42029.00         & 7.46    \\
            11 & lin318   & GA        & 429314     & 363904.18      & 42029.00         & 765.84  \\
            12 & rl5915   & GA        & 20712540   & 38904633.04    & 565530.00        & 6779.32 \\
            13 & rl5915   & GA2Opt    & 33367821   & 648205.16      & 565530.00        & 14.62   \\
            14 & u574     & GA        & 471427     & 457563.15      & 36905.00         & 1139.84 \\
            15 & u574     & GA2Opt    & 1260895    & 40516.22       & 36905.00         & 9.79    \\
            \bottomrule
        \end{tabular}
    \end{table}
\end{frame}

\section{Ant Colony System (ACS)}
\begin{frame}{Ant Colony System (ACS)}
    \textbf{Principio di Base}
    \begin{itemize}
        \item Ispirato al comportamento delle colonie di formiche nel cercare il cibo. Le formiche depositano una sostanza chimica (feromone) lungo il percorso mentre cercano il cibo, rafforzando i percorsi più brevi.
        \item Nel contesto del TSP, le formiche artificiali esplorano il grafo delle città e aggiornano i percorsi (archi) in base alla qualità della soluzione trovata.
    \end{itemize}

    \begin{table}[H]
        \centering
        \caption{Risultati ACS vs ACS+2Opt}
        \begin{tabular}{lllrrrr}
            \toprule
               & Istanza  & algorithm & Tempo (ms) & Lunghezza Tour & Lunghezza ottima & Gap   \\
            \midrule
            0  & berlin52 & ACS       & 18411      & 7606.66        & 7542.00          & 0.86  \\
            1  & berlin52 & ACS2Opt   & 20289      & 7598.44        & 7542.00          & 0.75  \\
            2  & d198     & ACS       & 94445      & 16584.44       & 15780.00         & 5.10  \\
            3  & d198     & ACS2Opt   & 101484     & 16079.65       & 15780.00         & 1.90  \\
            4  & eil76    & ACS2Opt   & 27703      & 558.76         & 538.00           & 3.86  \\
            5  & eil76    & ACS       & 33136      & 554.04         & 538.00           & 2.98  \\
            6  & fl1577   & ACS       & 2880286    & 25939.85       & 22249.00         & 16.59 \\
            7  & fl1577   & ACS2Opt   & 3852624    & 22914.07       & 22249.00         & 2.99  \\
            8  & lin105   & ACS2Opt   & 80683      & 14605.88       & 14379.00         & 1.58  \\
            9  & lin105   & ACS       & 123877     & 14981.78       & 14379.00         & 4.19  \\
            10 & lin318   & ACS2Opt   & 202670     & 43556.33       & 42029.00         & 3.63  \\
            11 & lin318   & ACS       & 338493     & 46646.52       & 42029.00         & 10.99 \\
            12 & rl5915   & ACS2Opt   & 34991642   & 612104.64      & 565530.00        & 8.24  \\
            13 & rl5915   & ACS       & 35872198   & 706188.19      & 565530.00        & 24.87 \\
            14 & u574     & ACS       & 452117     & 42512.54       & 36905.00         & 15.19 \\
            15 & u574     & ACS2Opt   & 490206     & 39051.51       & 36905.00         & 5.82  \\
            \bottomrule
        \end{tabular}
    \end{table}


\end{frame}


\section{Red-Black Ant Colony System (RB-ACS)}
\begin{frame}{Red-Black Ant Colony System (RB-ACS)}
    \textbf{Motivazioni e Introduzione}
    \begin{itemize}
        \item Variante migliorata dell'Ant Colony System (ACS) sviluppata per migliorare l’efficienza e la precisione.
        \item Introduce due gruppi di formiche (Rosse e Nere), ognuno dei quali lavora in parallelo su percorsi separati.
    \end{itemize}

    \begin{table}[H]
        \centering
        \caption{Risultati RB-ACS vs RB-ACS+2Opt}
        \begin{tabular}{lllrrrr}
            \toprule
               & Istanza  & algorithm & Tempo (ms) & Lunghezza Tour & Lunghezza ottima & Gap   \\
            \midrule
            0  & berlin52 & RBACS     & 34002      & 7681.45        & 7542.00          & 1.85  \\
            1  & berlin52 & RBACS2Opt & 34159      & 7544.37        & 7542.00          & 0.03  \\
            2  & d198     & RBACS     & 136998     & 17097.74       & 15780.00         & 8.35  \\
            3  & d198     & RBACS2Opt & 283417     & 16076.46       & 15780.00         & 1.88  \\
            4  & eil76    & RBACS     & 40567      & 562.41         & 538.00           & 4.54  \\
            5  & eil76    & RBACS2Opt & 40672      & 557.14         & 538.00           & 3.56  \\
            6  & fl1577   & RBACS     & 4210934    & 26788.92       & 22249.00         & 20.41 \\
            7  & fl1577   & RBACS2Opt & 6887456    & 23236.46       & 22249.00         & 4.44  \\
            8  & lin105   & RBACS     & 77720      & 14785.44       & 14379.00         & 2.83  \\
            9  & lin105   & RBACS2Opt & 262217     & 14489.08       & 14379.00         & 0.77  \\
            10 & lin318   & RBACS     & 469358     & 46823.70       & 42029.00         & 11.41 \\
            11 & lin318   & RBACS2Opt & 603140     & 43421.71       & 42029.00         & 3.31  \\
            12 & rl5915   & RBACS     & 21861319   & 716108.43      & 565530.00        & 26.63 \\
            13 & rl5915   & RBACS2Opt & 28403435   & 608329.94      & 565530.00        & 7.57  \\
            14 & u574     & RBACS     & 826684     & 43702.95       & 36905.00         & 18.42 \\
            15 & u574     & RBACS2Opt & 899785     & 39219.59       & 36905.00         & 6.27  \\
            \bottomrule
        \end{tabular}
    \end{table}
\end{frame}

\begin{frame}{Confronto degli Approcci al TSP}
    \small
    \begin{columns}[T,totalwidth=\textwidth]
        \begin{column}{0.5\textwidth}
            \textbf{Esatti}
            \begin{itemize}
                \item[+] Soluzione ottimale
                \item[-] Alta complessità
            \end{itemize}

            \textbf{Euristici}
            \begin{itemize}
                \item[+] Veloci
                \item[-] Non ottimali
            \end{itemize}

            \textbf{Simulated Annealing}
            \begin{itemize}
                \item[+] Evita ottimi locali
                \item[-] Convergenza lenta
            \end{itemize}
        \end{column}

        \begin{column}{0.5\textwidth}
            \textbf{Genetici}
            \begin{itemize}
                \item[+] Adattabili
                \item[-] Costo computazionale
            \end{itemize}

            \textbf{Ant Colony System}
            \begin{itemize}
                \item[+] Efficace su problemi complessi
                \item[-] Molti parametri
            \end{itemize}

            \textbf{Red-Black ACS}
            \begin{itemize}
                \item[+] Migliore esplorazione dello spazio
                \item[-] Maggiore complessità computazionale
            \end{itemize}
        \end{column}
    \end{columns}
\end{frame}


\section{Risultati Sperimentali}
\begin{frame}{Heatmap}
    \adjustbox{width=\paperwidth,height=\paperheight,keepaspectratio}{%
        \includegraphics{analysis/heatmap_full.pdf}%
    }
\end{frame}

\begin{frame}{Confronto degli Algoritmi}
    \adjustbox{width=0.90\paperwidth,height=\paperheight,keepaspectratio}{%
        \includegraphics{analysis/alg_comparison.pdf}%
    }
\end{frame}


\section{Conclusioni}
\begin{frame}{Conclusioni}
    \begin{itemize}
        \item Il Red-Black Ant Colony System (RB-ACS) ha dimostrato di essere un approccio promettente per la risoluzione del TSP e altri problemi di ottimizzazione combinatoria.
        \item Sebbene più complesso computazionalmente rispetto ad ACS, l'uso di due gruppi di formiche (Rosse e Nere) con parametri e strategie di aggiornamento differenziati consente una migliore esplorazione dello spazio delle soluzioni.
        \item Questo approccio ha portato a miglioramenti significativi nelle prestazioni, in particolare su istanze di grandi dimensioni.
        \item In alcuni casi, il RB-ACS ha superato gli approcci tradizionali, dimostrando il potenziale del suo design per affrontare le sfide computazionali dei problemi NP-hard.
        \item Studi futuri potrebbero estendere il RB-ACS ad altre varianti del TSP e ad altri problemi di ottimizzazione per verificare ulteriormente la sua efficacia.
    \end{itemize}
\end{frame}


\end{document}
